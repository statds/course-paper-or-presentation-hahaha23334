\documentclass{article}
\usepackage[utf8]{inputenc}
\usepackage{graphicx}
\usepackage{amsthm}
\usepackage[round]{natbib}
\usepackage{amsmath}
\usepackage{graphicx}
\usepackage{cite}
\usepackage{hyperref}
\usepackage[authoryear,round]{natbib}


\title{Applications of Number Theory in Statistics}
\author{
Ge Li \\
Department of Statistics \\
University of Connecticut}

\date{September 2023}

\begin{document}

\maketitle

\begin{abstract}
The integration of number theory with statistics has paved the way for significant advancements in various industries. Despite the widespread applications of number theory, its specific role in optimizing real-world challenges has often been overlooked. This paper delves into the innovative methodologies of Hua Luogeng, specifically the golden section optimization. Through a rigorous exploration, we elucidate the transformative impact of these strategies, from enhancing industrial practices to streamlining investment methodologies. By foregrounding the accessibility of these techniques, this study emphasizes their utility even for individuals without an in-depth mathematical background. This paper serves as a testament to the pragmatic and far-reaching influence of number theory in bridging the gap between complex mathematical challenges and their tangible real-world applications.
\end{abstract}

\section{Introduction}
In this paper, we explore the applications of Number Theory in the realm of Statistics. Number theory provides a fascinating toolset for addressing many statistical challenges. This exploration will be divided into several sections: section 2 will provide a background on the intertwined nature of number theory and statistics. Section 3 will delve deeper into specific techniques. In section 4, we discuss the broader implications and influences of this synergy. Section 5 will present other examples and applications. The appendix will offer the code used to generate our figures and tables.

\section{Background}
The relationship between number theory and statistics is profound and intricate. Whereas number theory revolves around the properties and relationships of numbers, especially the positive integers, statistics is concerned with data interpretation. However, many statistical methods rely on number theoretic concepts, like prime numbers, modular arithmetic, and congruences.

Consider the simple problem of understanding random distributions. One could apply number theoretic techniques to probe the "randomness" or predictability of certain distributions. Similarly, in cryptanalysis, number theory and statistics come together to decipher coded messages.

A relevant example is the application of the Chinese Remainder Theorem in understanding residues, which has implications in regression analysis.

\section{Techniques and Theories}
The primary techniques of number theory in statistics revolve around understanding data distributions, predicting patterns, and deciphering encoded messages. One of the widely acknowledged techniques involves the use of prime numbers.

Consider the distribution of prime numbers. The Prime Number Theorem gives an approximation for the number of primes less than a given value. This can be formulated as:
\begin{equation}
\pi(n) \approx \frac{n}{\log(n)}
\end{equation}
where \(\pi(n)\) denotes the number of primes less than \(n\).

Another pivotal concept is modular arithmetic, a system of arithmetic for integers where numbers wrap around after reaching a certain value—called the modulus. This can be represented as:
\begin{equation}
a \equiv b \mod{m}
\end{equation}
indicating that \(a\) and \(b\) give the same remainder when divided by \(m\).

\begin{figure}[h]
\centering
\includegraphics[scale=0.4]{prime_distribution.png}
\caption{Distribution of prime numbers}
\label{prime_distribution}
\end{figure}

Figure \ref{prime_distribution} showcases the distribution of prime numbers, a foundational topic in number theory, which has applications in randomness testing in statistics.


\section{Influence}
The integration of number theory in statistics has paved the way for innovative solutions in data analysis, cryptography, and even computer science. Through the lens of number theory, statisticians have gained a deeper understanding of data distributions, randomness, and patterns. This synergy has enriched both fields, leading to advancements that are crucial in the digital age\citet{wang2016}.

\section{Other Applications}
Beyond the aforementioned applications, number theory's principles find relevance in various statistical domains:

\begin{table}[h]
\centering
\begin{tabular}{|l|l|}
\hline
Application & Description \\
\hline
Cryptography & Using prime numbers for encoding \\
\hline
Randomness Testing & Using modular arithmetic \\
\hline
Error Detection & Via checksums based on remainders \\
\hline
\end{tabular}
\caption{Applications of Number Theory in Statistics}
\label{table:applications}
\end{table}

Table \ref{table:applications} enumerates some areas where number theoretic principles are applied in statistics.

Statistical tests, like the Chi-Squared test, often use number theory principles for computations. Furthermore, randomness tests, which evaluate the unpredictability of values, benefit from number theory's modular arithmetic\citet{latexcompanion}.

\section*{Appendix}
The code used to generate the figures and tables in this paper can be found in the accompanying code folder. Kindly refer to the README file in the code folder for execution instructions.


\bibliographystyle{plainnat}
\bibliography{bibliography}

\end{document}